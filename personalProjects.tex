%-------------------------------------------------------------------------------
%	SECTION TITLE
%-------------------------------------------------------------------------------
\cvsection{Personal Projects}


%-------------------------------------------------------------------------------
%	CONTENT
%-------------------------------------------------------------------------------
\begin{cventries}

%---------------------------------------------------------
  \cventry
    {Summarizing Software API Usage Examples using
Clustering Techniques} % Organization
	{}
	{}
    {Mar. 2016 - Present} % Date(s)
    {
      \begin{cvitems} % Description(s) of tasks/responsibilities
        \item {I'm the main developer behind \textit{CLAMS (Clustering for API Mining of Snippets)}, for mining API usage examples, a framework for mining API usage examples written in Python.}
		\item {The system clusters client usage examples based on their API calls, generates summarized versions for the top snippets of each cluster, and then selects the most representative snippet from each cluster, using a tree edit distance metric on the ASTs.}
		\item {This results in a set of high quality API usage examples in the form of concise and readable snippets, thus enabling and supporting source code reuse even in cases of libraries with sparse or minimal documentation.}
		\item {Method is entirely data-driven, requiring only syntactic information from the source code, and so is programming language- agnostic.}
		\item {This project started as an MSc Thesis project while being a postgraduate student at University of Edinburgh, being supervised by Prof. Charles Sutton and co-supervised by Mr Themistoklis Diamantopoulos (Aristotle University of Thessaloniki).}
		\item {I successfully applied CLAMS internally at Hotels.com during our innovation time and received encouraging feedback as part of the pilot study I conducted.}
		\item {The source code of the system is available at \url{https://github.com/mast-group/clams}}
      \end{cvitems}
    }
    
  \cventry
    {Mantissa: A Recommendation System for Test-Driven Code Reuse} % Organization
	{}
	{}
    {Jan. 2013 - Present} % Date(s)
    {
      \begin{cvitems} % Description(s) of tasks/responsibilities
        \item {Designed and developed \textit{Mantissa}, an RSSE (Recommendation System in Software Engineering) that allows code searching in growing repositories, using CSEs such as GitHub Search.} 
        \item{The back-end of Mantissa is written in Python and Java while the website is a Flask application where basic web development technologies like HTML, CSS and JavaScript have been used.}
        \item {Mantissa extracts the query from the source code of the developer, and employs AGORA and GitHub to search for available source code. The retrieved results are ranked using the Vector Space Model (VSM), while various Information Retrieval (IR) techniques and heuristics are employed in order to best rank the results.}
		\item {Additional source code transformations are performed so that each result can be ready-to-use by the developer. Apart from relevance scoring, each result is presented along with useful information indicating its original source code, its control flow, as well as any external dependencies.}
		\item {This project started as an MEng thesis project while being an undergraduate student at Aristotle University of Thessaloniki, being supervised by Prof. Andreas Symeonidis and co-supervised by Mr Themistoklis Diamantopoulos.}
		\item{During the last 2 years we've proceeded to several improvements and feature additions, including source code transformations and extraction of the source code flow of results' methods.}
		\item {The system is available at \url{http://mantissa.ee.auth.gr}}
      \end{cvitems}
    }
%---------------------------------------------------------
\end{cventries}
